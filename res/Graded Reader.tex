\documentclass{report}

% Add the ability to display 汉字 and English
\usepackage[UTF8]{ctex}
\usepackage[utf8]{inputenc}
% Add the ability to write pinyin with tones
\usepackage{xeCJK}
\usepackage{xpinyin}
% Add the ability to create fancy headers and footers
\usepackage{fancyhdr}
% Add the ability to use Lorem Ipsum placeholder text
\usepackage{lipsum}
\usepackage{xcolor}
% Add the ability to align text
\usepackage{setspace}
\usepackage{sectsty}
% Add the ability to underline
\usepackage[normalem]{ulem}
% Make "" switch to opposing quotation marks intelligently
\usepackage [english]{babel}
\usepackage [autostyle, english = american]{csquotes}
\MakeOuterQuote{"}
% Add the ability to give underlines a colour, and its thickness
\usepackage{color,soul}
\definecolor{darkblue}{rgb}{0,0.8,0.2}
\setulcolor{darkblue}
\setul{0.5ex}{0.4ex}
% Add the ability to use graphics
\usepackage{graphicx}
% Add the ability to indented paragraphs
\usepackage{indentfirst}
% Add the ability to use subscripts
\usepackage{fixltx2e}

\pagestyle{fancy}
\chapterfont{\centering}
%\fancyhf{}% Clear header/footer
%\fancyhead[C]{Header}
%\fancyfoot[C]{Footer}% \fancyfoot[R]{\thepage}
\renewcommand{\headrulewidth}{0pt}
\renewcommand{\footrulewidth}{0.4pt}
%\cfoot{}

%\lhead{}
%\chead{}
%\rhead{}
%\lfoot{}
%\rfoot{}
%\lhead{一本中文书}


%\renewcommand{\baselinestretch}{3} 

%\leftskip=0cm plus 0.5fil \rightskip=0cm plus -0.5fil
%\parfillskip=0cm plus 1fil

% Title Page
\title{James' Chinese Graded Reader LaTeX Template}
\author{Idios/JC}


%%%%% Begin Document
\begin{document}
	
%Make title page	
\maketitle
%Make info page
%\clearpage
%Why was this book written?
%This template was written to help contribute to the pool of Chinese graded readers, which allow readers to understand and follow a story written with simple language - with the aim of improving their reading ability whilst having a lot of fun!
%\clearpage
%Table of contents(characters,locations,chapters,key words...)


% Steps to making book:
% Write all the content, get all figures (can do writing even in notepad! focus on content creation!)
% Format your content with this template 
% Find and replace proper nouns w/ underlined versions
% Choose words to define, and do so by using e.g. \uline{字}\textsubscript{#}. Find and replace all instances of the character so all instances have the reference.
% For the first instance only, define (in this case) 字 in the footer

%Character page

\uline{name1} (\pinyin{li3shan1guai1}) description!

\uline{name2} (\pinyin{da4wei2}) description!

\uline{name3} (\pinyin{hong2yu4}) description!

\clearpage


%Page 1
\setstretch{1.5}
{\centering \LARGE
{Chapter 1}\\
{\uline{Chapter title?}}\\}



\setstretch{2.0}

{\Large "lorem ipsum?"lorem ipsum\ul{\mbox{对她说}}\textsubscript{x}。\\
\indent "你不知道\ul{啊}\textsubscript{x}?lorem ipsum\ul{\mbox{聪明}}\textsubscript{x}lorem ipsum!\ul{\mbox{比如}}\textsubscript{x},lorem ipsum,lorem ipsum\uline{李山乖}。"\\
\indent "lorem ipsum?"lorem ipsum\ul{\mbox{对她}}\textsubscript{x} ul{\mbox{再问}}\textsubscript{x}。\\
\indent lorem ipsum:"lorem ipsum。lorem ipsum?"\\
\indent lorem ipsum,lorem ipsum\uline{大卫}。



\indent "\uline{大卫},lorem ipsum!"lorem ipsum。可是,\uline{大卫}lorem ipsum,lorem ipsum。刚过他们,lorem ipsum。\uline{大卫}lorem ipsum,lorem ipsum。

\indent "lorem ipsum?!", lorem ipsum。\\
\indent "\ul{\mbox{谁知道}}。。。",lorem ipsum。} 

\lfoot{
	x. 对她说 (\pinyin{dui4ta1shuo1}) said to her\\
	x. 啊 (\pinyin{a1}) who knows..\\
	x. 聪明 (\pinyin{cong1ming}) intelligent\\
}
\rfoot{
	x. 比如 (\pinyin{bi3ru2}) for example\\
	x. 再问 (\pinyin{zai4wen4}) ask again\\
	x. 谁知道 (\pinyin{shei2zhi1dao}) who knows..?\\
}


\clearpage
% Page 2 

\begin{figure}[ht!]
	\centering
	\includegraphics[width=90mm]{exampleFigure.png}
%	\caption{A simple caption \label{overflow}}
\end{figure}

\lfoot{}
\rfoot{}
\renewcommand{\footrulewidth}{0pt}



\clearpage


\end{document}          
